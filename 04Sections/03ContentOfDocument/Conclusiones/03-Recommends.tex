\begin{itemize}
    \item Se sugiere obtener credenciales de acceso para el API Elsevier, con mayor capacidad de consultas, en especial para la obtención de información de Autores, que se hace mediante la API de Author Retrieval. Actualmente se cuenta con la limitación de 5000 consultas por semana, lo que puede ser ineficiente para la obtención de información de autores en grandes cantidades.
    \item Se recomienda mantener el enfoque de Arquitectura Hexagonal para futuras funcionalidades y mejoras en el sistema, ya que permite una fácil escalabilidad y mantenimiento del sistema.
    \item Se recomienda revisar constantemente la documentación de Neomodel, ya que al ser una herramienta relativamente nueva esta en constante actualización  y puede presentar cambios en su funcionamiento y en la forma de implementar ciertas funcionalidades.
    \item La funcionalidad de buscar Autores dado un query (nombre, apellido, ID, etc) en ocasiones no retorna resultados esperados, se recomienda revisar la lógica de búsqueda y la forma en que se realiza la consulta a la base de datos de Neo4j.
\end{itemize}