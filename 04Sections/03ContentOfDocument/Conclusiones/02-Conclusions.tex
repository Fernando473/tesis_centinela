El análisis de ResNet y ResearchDecide permitió identificar puntos clave en cada una de estas herramientas. 
A partir de este análisis, se definió una estrategia de integración que mantiene la independencia de cada sistema, mientras se facilita una comunicación eficiente entre las mismas. 
Esta estrategia asegura que ambos sistemas puedan colaborar sin comprometer su autonomía, optimizando el flujo de información y mejorando la eficacia global del proceso.



El uso de la Arquitectura Hexagonal en la versión 1 de Centinela, 
bajo el enfoque de `arquitectura limpia', 
permitió abordar de manera eficiente la separación de responsabilidades y la integración de componentes. 
Este enfoque facilitó la escalabilidad del sistema, mejoró la mantenibilidad del código y 
aseguró que los módulos centrales permanecieran independientes de las tecnologías externas,
permitiendo adaptaciones futuras sin afectar la lógica de negocio. 
Aunque su implementación requirió un mayor esfuerzo en la fase de diseño,
se logró una mayor eficiencia en la fase de desarrollo.

La integración automatizada vía API con Scopus ha convertido a Centinela en una herramienta más robusta y completa. Esta integración permite acceder a la información más actualizada y confiable de la base de datos de Scopus. Gracias a este proceso automatizado, un administrador puede extraer información directamente desde Scopus, asegurando que la base de datos de Centinela se mantenga siempre actualizada de manera eficiente y sin intervención manual.
Además que se garantiza la calidad, integridad y disponibilidad de los datos, lo que mejora la confiabilidad de los resultados obtenidos por los usuarios de Centinela.