\subsection{Introducción}
El cuarto sprint tiene como objetivo alcanzar el estado final de Resnet,
es decir, lograr que Centinela esté completamente funcional y lista para ser
utilizada por los usuarios. En el contexto del motor de búsqueda,
se espera que  Centinela sea capaz de realizar las búsquedas
que Resnet podía realizar en su estado final, alcanzando así el nivel de funcionalidad de la versión anterior.
\subsection{Objetivos}
\begin{itemize}
    \item Implementar la automatización para generar el Corpus y el Modelo de TF-IDF
    \item Implementar la funcionalidad de búsqueda de autores relevantes dado un topico
    \item Implementar la funcionalidad de búsqueda de articulos relevantes dado un topico
    \item Implementar la funcionalidad para obtener la red de coautoría de un autor
    \item Implementar la funcionalidad para obtener un Articulo dado su ID de Scopus
    \item Implementar la funcionalidad para obtener un Autor dado su ID de Scopus
\end{itemize}
\subsection{Planificación}
Para este sprint se terminaron las tareas que quedaron pendientes en el Sprint 1 (Ver sección \ref{chapter02-section02-sprint1})  y en el Sprint 2 (Ver sección \ref{chapter02-section02-sprint2}).
En la tabla \ref{C2T4:Historias de Usuario del Sprint 4} se presentan las historias de usuario que se abordarán en este sprint con sus respectivas tareas.
\begin{table}[!t]
    \centering
    \begin{tabular}{|p{2.5cm}|p{5cm}|p{6cm}|}
        \midrule
        \textbf{Identificador} & \textbf{Historia de Usuario}                                                                                                                                                                               & \textbf{Tareas} \\
        \hline
        HU-SE-01 & Como usuario no registrado deseo poder encontrar artículos relevantes dado un  tema de investigación para poder acceder rápidamente a información útil y actualizada que apoye mi estudio o trabajo        &
        \begin{compactitem}
            \item Generar el corpus \footnote{
                Conjunto de documentos que se utilizan para entrenar un modelo de \textit{Machine Learning} o \textit{Deep Learning}
            } de datos  
            \item Generar el modelo de TF-IDF a partir del corpus
            \item Crear el modelo de Artículo para la base de datos
            \item Implementar el filtro de artículos por año
            \item Crear un servicio de búsqueda de artículos
            \item Desarrollar un endpoint para manejar las solicitudes entrantes
        \end{compactitem}
        \\
        \hline
        HU-SE-04 & Como usuario no registrado quiero poder ver los artículos de un investigador para conocer su trabajo y las publicaciones en las que ha contribuido &
        \begin{compactitem}
            \item Implementar un endpoint para recuperar el perfil de un investigador y sus publicaciones
            \item Crear un servicio de búsqueda de investigadores y sus artículos
        \end{compactitem}
        \\
        \hline
        HU-SE-03 & Como usuario no registrado, deseo poder ver la red de coautoría de un autor para visualizar un grafo con los autores con los que ha colaborado, así como la fuerza de esas colaboraciones &
        \begin{compactitem}
            \item Implementar un endpoint para recuperar los datos de colaboración entre autores
            \item Crear un servicio que genere los datos del grafo de coautoría de un autor
        \end{compactitem}
        \\
        \hline
    \end{tabular}
    \caption{Historias de Usuario del sprint 4}
    \label{C2T4:Historias de Usuario del Sprint 4}
\end{table}


Los criterios de aceptación para las historias de usuario HU-SE-01, HU-SE-03 y HU-SE-04 han sido definidos y se pueden consultar en las siguientes figuras: Figura \ref{fig:aceptance-criteria-HU-SE-01}, Figura \ref{C2F2:Criterios de Aceptacion HU-SE-03} y Figura \ref{C2F2:Criterios de Aceptacion HU-SE-04}.
\subsection{Implementación}

Para la implementación de ciertas tareas en este Sprint, se optó por reutilizar el código de la versión anterior de ResNet, ya que se consideró más eficiente y rápido que comenzar desde cero.
La optimización de funcionalidades existentes no está incluida en el alcance de este componente.
Las funcionalidades que se reutilizarán son la generación del corpus y el modelo de TF-IDF,
dado que estas tareas son las más demandantes en términos de tiempo y se determinó que no era necesario reimplementarlas.
A estas funcionalidades solo se les añadirá una capa de infraestructura,
ya que la lógica de negocio ya está implementada en la versión anterior de ResNet.
Esto garantizará un punto de entrada para generar tanto el corpus como el modelo de TF-IDF,
permitiendo su utilización en las búsquedas de autores y artículos.

Para la tarea \textit{HU-SE-01: Implementar la automatización para generar el Corpus y el Modelo de TF-IDF}, 

\subsection{Revisión y Retrospectiva}
