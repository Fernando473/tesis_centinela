
% \section*{Historias de Usuario para ResNet}

% \subsection*{1. Como administrador de ResNet, quiero tener la capacidad de agregar, modificar y eliminar usuarios, así como asignar roles y permisos, para mantener el control de la accesibilidad y garantizar que cada usuario tenga la autorización adecuada.}

% \subsubsection*{Descripción}
% El administrador debe poder gestionar las cuentas de usuario dentro de ResNet para garantizar un acceso seguro y autorizado al sistema.

% \subsubsection*{Criterios de Aceptación}
% \begin{itemize}
%   \item Debe haber una interfaz de administración fácil de usar para gestionar usuarios.
%   \item El administrador puede agregar nuevos usuarios proporcionando información relevante y asignando roles.
%   \item Se debe permitir la modificación de información de usuario y la actualización de roles y permisos.
%   \item La capacidad de eliminar cuentas de usuario también debe estar disponible para el administrador.
% \end{itemize}

% \subsection*{2. Como administrador de ResNet, quiero tener la capacidad de extraer los datos de SCOPUS de manera automática, para mantener el sistema actualizado.}

% \subsubsection*{Descripción}
% El administrador necesita una funcionalidad automatizada para extraer datos de SCOPUS y actualizar la base de datos de ResNet de manera periódica.

% \subsubsection*{Criterios de Aceptación}
% \begin{itemize}
%   \item Se debe integrar una función de extracción automática en el sistema.
%   \item Debe ser posible configurar la frecuencia de las extracciones automáticas.
%   \item El sistema debe manejar errores de extracción y notificar al administrador si hay problemas.
%   \item La información extraída debe actualizarse en la base de datos de ResNet de manera coherente.
% \end{itemize}

% \subsection*{3. Como administrador de ResNet, quiero poder autenticarme para ingresar al panel de administrador.}

% \subsubsection*{Descripción}
% El administrador debe tener un proceso de autenticación seguro y confiable para acceder al panel de administración de ResNet.

% \subsubsection*{Criterios de Aceptación}
% \begin{itemize}
%   \item Se debe implementar un sistema de inicio de sesión que requiera credenciales válidas.
%   \item La autenticación debe utilizar medidas de seguridad, como el cifrado de contraseñas.
%   \item Deben existir mecanismos de recuperación de contraseña para garantizar el acceso en caso de olvido.
%   \item El sistema debe bloquear temporalmente el acceso después de varios intentos fallidos para prevenir ataques de fuerza bruta.
% \end{itemize}


