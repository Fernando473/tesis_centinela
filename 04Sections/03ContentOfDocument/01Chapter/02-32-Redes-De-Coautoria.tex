Las redes de coautoría se construyen a partir de las publicaciones científicas, donde cada autor se representa como un nodo en la red y las colaboraciones entre autores se representan como enlaces entre los nodos correspondientes. Estas redes proporcionan una representación visual y matemática de las relaciones de colaboración en la comunidad científica. Al analizar estas redes, es posible identificar la estructura de la colaboración, encontrar comunidades de autores que colaboran estrechamente, identificar autores influyentes y recomendar nuevas colaboraciones potenciales.

Además, las redes de coautoría se utilizan para estudiar tendencias en la producción científica, identificar áreas de investigación interconectadas y evaluar la difusión del conocimiento en diferentes campos. En resumen, estas redes proporcionan una herramienta poderosa para comprender la dinámica de la colaboración científica y para facilitar la identificación de oportunidades de colaboración entre investigadores.