TF-IDF (Term Frequency-Inverse Document Frequency) es una técnica de procesamiento de lenguaje natural que se utiliza para evaluar la relevancia de un término en un documento o corpus de documentos. Esta técnica se utiliza comúnmente en la recuperación de información y la minería de texto.

La técnica TF-IDF se basa en dos conceptos principales: la frecuencia del término (TF) y la frecuencia inversa del documento (IDF). La frecuencia del término se refiere al número de veces que un término aparece en un documento, mientras que la frecuencia inversa del documento se refiere a la frecuencia con la que aparece un término en todo el corpus de documentos.

La fórmula para calcular el valor TF-IDF de un término es:

\(\text{TF-IDF} = \text{TF} \cdot \log\left(\frac{N}{\text{DF}}\right)\)



Donde TF es la frecuencia del término en el documento, N es el número total de documentos en el corpus y DF es el número de documentos que contienen el término.

El valor TF-IDF de un término es alto si aparece con frecuencia en un documento específico, pero rara vez aparece en otros documentos del corpus. Esto indica que el término es importante y relevante para el documento en cuestión.

La técnica TF-IDF se utiliza comúnmente en la recuperación de información para clasificar y ordenar documentos según su relevancia para una consulta de búsqueda específica. También se utiliza en la minería de texto para identificar patrones y tendencias en grandes conjuntos de datos de texto.