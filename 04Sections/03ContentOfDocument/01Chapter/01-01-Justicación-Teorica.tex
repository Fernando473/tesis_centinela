La creciente producción científica a nivel mundial y la necesidad de colaboración entre investigadores demandan herramientas innovadoras para facilitar la conexión y el intercambio de conocimiento. Centinela, una plataforma que integra ResNet y ResearchDecide, responde a esta necesidad al proporcionar una visión integral de las redes de coautoría en Ecuador y al facilitar la formación de nuevos equipos de investigación.

Basándonos en la teoría de redes sociales \cite{REDES-SOCIALES-TEORIA}, planteamos que la estructura de las redes de coautoría influye significativamente en la productividad científica y en la difusión del conocimiento. Al analizar estas redes, Centinela permitirá identificar patrones de colaboración, detectar vacíos de conocimiento y facilitar la formación de nuevos grupos de investigación.

La automatización del proceso de extracción de datos, fundamentada en los principios de la minería de datos, garantizará la calidad y la consistencia de la información, lo que a su vez permitirá realizar análisis más precisos y confiables. En última instancia, Centinela contribuirá a fortalecer el sistema científico y a fomentar la colaboración en el Ecuador.