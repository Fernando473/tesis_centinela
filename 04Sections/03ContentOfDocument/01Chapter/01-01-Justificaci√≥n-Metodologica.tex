SCRUM se utilizará como la metodología principal para este proyecto debido a su capacidad para gestionar el desarrollo de software de manera iterativa y adaptativa. SCRUM facilita el trabajo en ciclos cortos, conocidos como sprints, que permiten la entrega continua de valor y una rápida adaptación a los cambios en los requisitos. Esto es especialmente útil en entornos dinámicos, donde los requisitos pueden evolucionar con frecuencia.

SCRUM proporciona visibilidad constante del progreso del proyecto y fomenta la colaboración y comunicación entre los miembros del equipo y los stakeholders. Su enfoque en la mejora continua y la gestión temprana de riesgos ayuda a identificar y solucionar problemas de manera proactiva. Al integrar SCRUM en el proyecto, se asegura una mayor eficiencia en la entrega de productos funcionales y una respuesta efectiva a los desafíos emergentes.