
En respuesta al aumento de la producción científica y a la creciente necesidad de colaboración entre expertos, se desarrolló la aplicación web ResNet. Esta herramienta utiliza modelos de minería de datos para buscar y mostrar redes de investigadores vinculados a instituciones ecuatorianas y sus áreas académicas.


El objetivo principal de la herramienta es presentar de manera visual las redes de coautoría entre investigadores ecuatorianos, utilizando datos extraídos de Scopus. 

Asimismo, ResNet facilita la búsqueda a través de diversos filtros, permitiendo a los usuarios afinar sus consultas de manera eficiente y personalizada.

En la actualidad, la aplicación ResNet se encuentra restringida a la extracción manual de datos desde Scopus, lo que ha evidenciado una necesidad apremiante de automatizar este proceso. Esta necesidad se aborda mediante la implementación de la metodología CRISP-DM (CRoss-Industry Standard Process for Data Mining), que proporciona directrices sólidas para la automatización eficiente de la extracción y preprocesamiento de datos, contribuyendo así a mejorar la eficacia y la velocidad del sistema. Este enfoque permite optimizar la recopilación de información desde Scopus, agilizando el flujo de trabajo y proporcionando resultados más precisos y oportunos en la construcción de redes de investigadores y coautoría.