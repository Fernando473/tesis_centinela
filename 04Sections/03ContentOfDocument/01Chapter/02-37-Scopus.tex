Scopus es una destacada base de datos bibliográfica y herramienta de indexación científica ampliamente utilizada en el ámbito académico y de la investigación. Desarrollada por la editorial Elsevier, Scopus abarca diversas disciplinas académicas, incluyendo ciencias sociales, ciencias de la salud, ingeniería, ciencias naturales y ciencias exactas.

Esta plataforma proporciona un extenso repositorio de información, que incluye resúmenes y citas de artículos científicos publicados en revistas revisadas por expertos, conferencias y libros. Su alcance internacional abarca publicaciones de diversas partes del mundo, lo que la convierte en una herramienta valiosa para investigadores que buscan acceder a una amplia gama de literatura científica.

Scopus no solo indexa la información, sino que también ofrece herramientas analíticas que permiten evaluar el impacto de las publicaciones y la productividad de los investigadores. Asimismo, su capacidad para rastrear las citas entre artículos facilita el seguimiento del desarrollo de las investigaciones y la identificación de tendencias en diversas áreas del conocimiento.