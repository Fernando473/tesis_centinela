Al iniciar un proyecto, es crucial que el equipo involucrado esté al tanto de sus roles, actividades y los plazos para la entrega de resultados. Scrum es un marco de trabajo que facilita este proceso, permitiendo además acelerar la entrega de valor al cliente a través de iteraciones cortas \cite{SCRUM}.

En 2001, un grupo de desarrolladores en Salt Lake City creó el Manifiesto Ágil, que se resume en los siguientes cuatro valores:

\begin{enumerate}
    \item Valorar a los individuos y su interacción por encima de los procesos y herramientas.
    \item Valorar el software que funciona por encima de la documentación exhaustiva.
    \item Valorar la colaboración con el cliente por encima de la negociación contractual.
    \item Valorar la respuesta al cambio por encima del seguimiento de un plan.
\end{enumerate}

El marco de trabajo SCRUM, dentro del contexto del Manifiesto Ágil, se fundamenta en aspectos como la flexibilidad, el factor humano, la colaboración y el desarrollo iterativo, con el objetivo principal de asegurar buenos resultados \cite{SCRUM}.
Scrum  posee herramientas que permiten resolver problemas y mejorar la administración del proyecto, los cuales son:
\begin{itemize}
    \item \textbf{Product Backlog}: Lista dinámica de características, requisitos, arreglos y mejoras que se deben completar.
    \item \textbf{Sprint Backlog}: Lista de elementos que el equipo de desarrollo debe completar durante el tiempo definido del sprint \cite{SCRUM-Sprints}.
    \item \textbf{Incremento}: Entregable que surge al final del sprint.
\end{itemize}


Scrum define tres roles principales: Product Owner, Scrum Master y Team. Estos roles son esenciales en cada equipo, que se caracterizan por ser auto-organizados y multifuncionales \cite{SCRUM-Roles}.

\begin{itemize}
    \item \textbf{Product Owner}: Encargado de maximizar el valor del trabajo y el perfil que habla con el cliente.
    \item \textbf{Scrum Master}: Responsable de que las técnicas y eventos de scrum sean aplicadas.
    \item \textbf{Team}:  Es un grupo de profesionales que trabajan juntos para entregar un incremento de producto terminado al final de cada sprint. El equipo de desarrollo es auto-organizado y multidisciplinario, lo que significa que posee todas las habilidades necesarias para crear el producto sin depender de personas externas al equipo.
\end{itemize}