Las bases de datos orientadas a grafos son sistemas de gestión de bases de datos diseñados específicamente para almacenar, recuperar y gestionar datos que se pueden representar como grafos. En lugar de utilizar el modelo de datos relacional tradicional, las bases de datos orientadas a grafos utilizan estructuras de datos de grafo para representar y almacenar la información.

En un grafo, los datos se modelan como nodos (también conocidos como vértices) que están conectados por relaciones (también conocidas como aristas). Cada nodo y relación pueden contener propiedades que describen los atributos de los datos. Este enfoque es especialmente útil para modelar y consultar datos que tienen relaciones complejas y que requieren un análisis de redes.

Las bases de datos orientadas a grafos son ampliamente utilizadas en una variedad de aplicaciones, incluyendo redes sociales, recomendaciones personalizadas, análisis de redes, sistemas de recomendación, bioinformática, gestión del conocimiento y análisis de datos interconectados. Algunas bases de datos orientadas a grafos populares incluyen Neo4j, Amazon Neptune, y JanusGraph, entre otras.